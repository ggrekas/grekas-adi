%%This is a very basic article template.
%%There is just one section and two subsections.

\documentclass[12pt]{article}
\usepackage{fullpage}

\usepackage[english]{babel}
\usepackage[utf8x]{inputenc}
\usepackage{amsfonts}
\usepackage{amsmath}
\usepackage{enumerate}


\begin{document}

\section{Introduction}
Let $\Omega=(a_x, b_x)(a_y, b_y)\in \mathbb{R}^2$ with its boundary $\Gamma=\partial \Omega$ and
$J=(0,T]$ be the time interval, $T>0$. The following Initial Boundary Value Problem (IBVP) is considered:
\begin{eqnarray}
u_t= \nabla\cdot(a\nabla u) + c(t, x, y) u + f&, \quad t\in J, \quad (x,y)\in \Omega \label{eq:heat}\\
u(x,y,0) = u_0(x,y)&\\
u_x(0,y,t)= g_1(y,t), \quad u_x(1,y,t)=g_2(y,t)&\\
u_y(x,0,t)= h_1(x,t), \quad u_y(x,1,t)= h_2(x,t)& 
\end{eqnarray}




\section{ADI Method implementation.}
 Alternating Direction Implicit method is a computationally efficient scheme,
 that satisfies the following properties:
  \begin{itemize}
    \item has accuracy $O(k^2 + h^2)$, $k$ is the time step and $h$ the spatial
    step;
    \item is unconditionally stable;
    \item number of operations per time step is proportional to the number of unknowns, $O(M)$
     where $ M$ is the number of unknowns.       
  \end{itemize}

\subsection{Douglas-Peaceman-Rachford method}
Let us define the operators:
$$A_1u = -(au_x)_x - \frac{1}{2}cu, \quad A_2u = -(au_y)_y - \frac{1}{2}cu,$$
then equation (\ref{eq:heat}) can be rewitten as:
\begin{eqnarray} 
u_t + A_1u + A_2u = f. \label{eq:heatA}
\end{eqnarray}
Replacing the operators $A_1$ and $A_2$ (\ref{eq:heatA}) respectively by their spatial approximations
$A_{1h}$ and $A_{2h}$, the proposed Douglas-Peaseman-Rachford method is:
\begin{eqnarray}
(I+\frac{k}{2}A_{1h})u^* = (I-\frac{k}{2}A_{2h})u^n + \frac{k}{2}f^{n+1/2}, \quad \text{(x-sweep)} 
\label{equ:x-sweep}\\
(I+\frac{k}{2}A_{2h})u^{n+1} = (I-\frac{k}{2}A_{1h})u^* + \frac{k}{2}f^{n+1/2}, \quad \text{(y-sweep)}
\end{eqnarray} \label{equ:y-sweep}

In more detail the left hand side of equation (\ref{equ:x-sweep}) is:
\begin{align*}
(I +\frac{k}{2}A_{1h})u^* &= u^* -\frac{k}{2}\Bigl( a_x u^*_x + a u^*_{xx} +
\frac{1}{2} c u^*\Bigr) \nonumber \\
&\simeq  u^*(x,y) -\frac{k}{2}\Bigl( \frac{a(x+h, y) - a(x-h,y)}{2h}
\frac{u^*(x+h, y)- u^*(x-h,y)}{2h} \\
& + a(x,y) \frac{u^*(x+h,y) - 2 u(x,y) + u^*(x-h,y) }{h^2}
+ \frac{1}{2} c(x,y) u^*(x,y)
\Bigr)
\end{align*}



Equation (\ref{equ:x-sweep}) results in the following linear system:
\begin{eqnarray}
  A_l u_{;,l} = B_l
  \label{equ:xsweep_lin}
\end{eqnarray}



with
 
\[ A_l = \left( \begin{array}{cccccc}
D_{1,l} & H_{1,l} & 0 & . & . & 0\\
S_{1,l} & D_{2,l} & H_{2,l} & 0 & . & 0 \\
. & . & . & . & .& . \\
0 & . & 0 & S_{N-3,l} & D_{N-2,l} & H_{N-2,l}\\
0 & . & . & 0 & S_{N-2,l} & D_{N-1,l}\\
\end{array} \right)
%
u_{;,l} = \left( \begin{array}{c}
  u_{1,l} \\
  u_{2,l} \\
  \vdots \\
  u_{N-1,l}
   \end{array}
\right)
\]


where $D_{;,l}$, $H_{;,l}$ and $S_{;,l}$ are defined as follows respectively: 
\[ D_{;,l} = \left( \begin{array}{c}
  \frac{1}{k} + \frac{a_{1,l}}{h^2} - 0.25 C_{1,l} \\
  \frac{1}{k} + \frac{a_{2,l}}{h^2} - 0.25 C_{2,l} \\
  \vdots \\
  \frac{1}{k} + \frac{a_{N,l}}{h^2} - 0.25 C_{N,l}
  \end{array} \right),
%
 H_{;,l} = \frac{1}{2}\left( \begin{array}{c}
  -2\frac{a_{1,l}}{h^2} \\
  -\frac{a^x_{2,l}}{2h} - \frac{a_{2,l}}{h^2} \\
  \vdots \\
  -\frac{a^x_{N-1,l}}{2h} - \frac{a_{N-1,l}}{h^2}
   \end{array}
\right),  
%
S_{;,l} = \frac{1}{2} \left( \begin{array}{c}
  \frac{a^x_{2,l}}{2h} - \frac{a_{2,l}}{h^2} \\
  \vdots \\
  \frac{a^x_{N-1,l}}{2h} - \frac{a_{N-1,l}}{h^2} \\
  -2\frac{a_{N, l}}{h^2}
  
   \end{array}
\right)\]

For the right hand side of equation (\ref{equ:xsweep_lin}) we have:
\begin{eqnarray}
\bigl(B_{l} \bigr)_i &=& \bigl(G_{l} \bigr)_i + \frac{1}{2}f^{n+1/2}_{i,l}\\
\bigl(G_{l} \bigr)_i &=&
\frac{1}{2}\Bigl[
\Bigl(-\frac{a^y_{i,l}}{2h} + \frac{a_{i,l}}{h^2}\Bigr) u^n_{i,l-1} +
\Bigl(\frac{a^y_{i,l}}{2h} + \frac{a_{i,l}}{h^2} \Bigr) u^n_{i,l+1}
\Bigr] + 
\Bigl(\frac{1}{k} - \frac{ a_{i,l}}{h^2} + \frac{1}{4}C_{i,l}\Bigr) u^n_{i,l} \\
 & &2 < l < L-1 \nonumber \\
\bigl(G_{1} \bigr)_i &=& \frac{a_{i,1}}{h^2} u^n_{i,2} + 
\Bigl(\frac{1}{k} - \frac{ a_{i,1}}{h^2} + \frac{1}{4}C_{i,1}\Bigr) u^n_{i,1} \\
% %
\bigl(G_{L} \bigr)_i &=&
\frac{a_{i,L}}{h^2} u^n_{i,L-1} + 
\Bigl(\frac{1}{k} - \frac{ a_{i,L}}{h^2} + \frac{1}{4}C_{i,L}\Bigr) u^n_{i,L}
\end{eqnarray}


\subsection{Chemotaxis addition.}
Let $\Omega=(a_x, b_x)\times(a_y, b_y) \in \mathbb{R}^2$ with its boundary
$\Gamma=\partial \Omega$ and $J=(0,T]$ be the time interval, $T>0$. The
following Initial Boundary Value Problem (IBVP) is considered:
\begin{eqnarray}
u_t= \nabla\cdot(a\nabla u) + c_g\nabla\cdot(u \nabla g) +  \nonumber \\
a_h\nabla\cdot(u \nabla h) +c(t, x, y) u + f&, \quad t\in J, &\quad 
(x,y)\in \Omega \label{eq:heatChem}\\ 
u(0,x,y) = u_0(x,y) \nonumber \\ 
u_x=u_y= 0,&  &\quad (x,y) \in \Gamma \nonumber \\ 
g_x=g_y= 0, & &\quad (x,y) \in\Gamma \nonumber \\ 
h_x=h_y= 0, & &\quad (x,y) \in\Gamma \nonumber
\end{eqnarray}
Defining the operatos:
\begin{eqnarray}
A_1u = -(au_x)_x  - c_g(u g_x)_x - c_h(u h_x)_x  - \frac{1}{2}cu, \nonumber \\
A_2u = -(au_y)_y  - c_g(u g_y)_y - c_h(u h_y)_y  - \frac{1}{2}cu, \nonumber \\
\end{eqnarray}
x-sweep linear system's elements are altered as follows:
\[ D_{;,l} = \left( \begin{array}{c}
  \frac{1}{k} + \frac{a_{1,l}}{h^2} - 0.25 C_{1,l} - (c_{bx})_{1,l} \\
  \frac{1}{k} + \frac{a_{2,l}}{h^2} - 0.25 C_{2,l} - (c_{bx})_{2,l}\\
  \vdots \\
  \frac{1}{k} + \frac{a_{N,l}}{h^2} - 0.25 C_{N,l} - (c_{bx})_{N,l}
  \end{array} \right),
\]
%
\[
 H_{;,l} = \frac{1}{2}\left( \begin{array}{c}
  -2\frac{a_{1,l}}{h^2} \\
  -\frac{a^x_{2,l} + (c_{ax})_{2,l}}{2h} - \frac{a_{2,l}}{h^2} \\
  \vdots \\
  -\frac{a^x_{N-1,l} + (c_{ax})_{N-1,l}}{2h} - \frac{a_{N-1,l}}{h^2}
   \end{array}
\right),  
%
S_{;,l} = \frac{1}{2} \left( \begin{array}{c}
  \frac{a^x_{2,l} + (c_{ax})_{2,l}}{2h} - \frac{a_{2,l}}{h^2} \\
  \vdots \\
  \frac{a^x_{N-1,l} + (c_{ax})_{N-1,l}}{2h} - \frac{a_{N-1,l}}{h^2} \\
  -2\frac{a_{N, l}}{h^2}
  
   \end{array}
\right)\]
where
\begin{eqnarray}
 (c_{ax})_{i,l} = (c_g)_{i,l} (g_x)_{i,l} + (c_h)_{i,l} (h_x)_{i,l}, \: 
 (c_{bx})_{i,l} = (c_g)_{i,l} (g_{xx})_{i,l} + (c_h)_{i,l} (h_{xx})_{i,l} 
\end{eqnarray}


For the right hand side of equation (\ref{equ:xsweep_lin}) we have:
\begin{eqnarray}
\bigl(B_{l} \bigr)_i &=& \bigl(G_{l} \bigr)_i + \frac{1}{2}f^{n+1/2}_{i,l}\\
\bigl(G_{l} \bigr)_i &=&
\frac{1}{2}\Bigl[
\Bigl(-\frac{a^y_{i,l} -(c_{ay})_{i,l}}{2h} + \frac{a_{i,l} + }{h^2}\Bigr)
u^n_{i,l-1} +
\Bigl(\frac{a^y_{i,l} + (c_{ay})_{i,l}}{2h} + \frac{a_{i,l}}{h^2} \Bigr)
u^n_{i,l+1}
\Bigr]\\
 &+& \Bigl(\frac{1}{k} - \frac{ a_{i,l}}{h^2} + \frac{1}{4}C_{i,l} +
 (c_{by})_{i,l} \Bigr) u^n_{i,l} \nonumber \\
 & &2 < l < L-1 \nonumber \\
\bigl(G_{1} \bigr)_i &=& \frac{a_{i,1}}{h^2} u^n_{i,2} + 
\Bigl(\frac{1}{k} - \frac{ a_{i,1}}{h^2} + \frac{1}{4}C_{i,1} + (c_{by})_{i,l}
\Bigr) u^n_{i,1} \\
%
\bigl(G_{L} \bigr)_i &=&
\frac{a_{i,L}}{h^2} u^n_{i,L-1} + 
\Bigl(\frac{1}{k} - \frac{ a_{i,L}}{h^2} + \frac{1}{4}C_{i,L}\Bigr) u^n_{i,L}
\end{eqnarray}
where
\begin{eqnarray}
 (c_{ay})_{i,l} = (c_g)_{i,l} (g_y)_{i,l} + (c_h)_{i,l} (h_y)_{i,l}, \: 
 (c_{by})_{i,l} = (c_g)_{i,l} (g_{yy})_{i,l} + (c_h)_{i,l} (h_{yy})_{i,l} 
\end{eqnarray}



{\LaTeX}


\end{document}
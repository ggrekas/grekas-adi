%%This is a very basic article template.
%%There is just one section and two subsections.

\documentclass[12pt]{article}
\usepackage{fullpage}

\usepackage[english]{babel}
\usepackage[utf8x]{inputenc}
\usepackage{amsfonts}
\usepackage{amsmath}
\usepackage{enumerate}


\begin{document}

\section{Introduction}
Let $\Omega=(a_x, b_x)(a_y, b_y)\in \mathbb{R}^2$ with its boundary $\Gamma=\partial \Omega$ and
$J=(0,T]$ be the time interval, $T>0$. The following Initial Boundary Value Problem (IBVP) is considered:
\begin{eqnarray}
u_t= \nabla\cdot(a\nabla u) + c(t, x, y) u + f&, \quad t\in J, \quad (x,y)\in \Omega \label{eq:heat}\\
u(x,y,0) = u_0(x,y)&\\
u_x(0,y,t)=u_x(1,y,t)=0&\\
u_y(x,0,t)=u_y(x,1,t)=0& 
\end{eqnarray}




\section{ADI Method implementation.}
 Alternating Ditection Implicit method is a computationally efficient scheme, that satisfies the
  following properties:
  \begin{itemize}
    \item has accurancy $O(k^2 + h^2)$, $k$ is the time step and $h$ the spatial step;
    \item is unconditionally stable;
    \item number of operations per time step is proportional to the number of unknowns, $O(M)$
     where $M$ is the number of unknowns.       
  \end{itemize}

\subsection{Douglas-Peaceman-Rachford method}
Let us define the operators:
$$A_1u = -(au_x)_x - \frac{1}{2}cu, \quad A_2u = -(au_y)_y - \frac{1}{2}cu,$$
then equation (\ref{eq:heat}) can be rewitten as:
\begin{eqnarray} 
u_t + A_1u + A_2u = f. \label{eq:heatA}
\end{eqnarray}
Replacing the operators $A_1$ and $A_2$ (\ref{eq:heatA}) respectively by their spatial approximations
$A_{1h}$ and $A_{2h}$, the proposed Douglas-Peaseman-Rachford method is:
\begin{eqnarray}
(I+\frac{k}{2}A_{1h})u^* = (I-\frac{k}{2}A_{2h})u^n + \frac{k}{2}f^{n+1/2}, \quad \text{(x-sweep)} 
\label{equ:x-sweep}\\
(I+\frac{k}{2}A_{2h})u^{n+1} = (I-\frac{k}{2}A_{1h})u^* + \frac{k}{2}f^{n+1/2}, \quad \text{(y-sweep)}
\end{eqnarray} \label{equ:y-sweep}

 
\subsection{Another subtitle}

{\LaTeX}


\end{document}